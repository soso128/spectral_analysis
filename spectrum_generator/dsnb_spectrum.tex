\documentclass[12pt]{article}
\usepackage[margin=1.1in]{geometry}
\usepackage{graphicx}
\usepackage{booktabs}
\usepackage{amsmath}
\usepackage{hyperref}

\title{DSNB spectrum parameterization}
\author{Sonia El Hedri}

\begin{document}
\maketitle
This note puts together some of the main parameterizations of the Diffuse Supernova Neutrino Background (DSNB) encountered in the literature. The total flux from the DSNB can be written as follows:
\begin{align}
    \Phi(E_\nu) = \frac{c}{H_0}\int \sum_{s}  R_{SN}(z, s)\, \sum_{\nu_i,\bar\nu_i} F_{i}\left[E_\nu (1 + z), s\right] \times \frac{dz}{\sqrt{\Omega_M(1+z)^3 + \Omega_\Lambda}}.
    \label{eq:dsnb}
\end{align}
Here, $c$ is the speed of light, $H_0$ is the Hubble constant today, $z$ is the redshift, $R_{SN}$ is the core-collapse supernova rate, $F_i$ is the supernova emission spectrum for a given flavor of (anti)neutrino, and $\sqrt{\Omega_M(1+z)^3 + \Omega_\Lambda}$ is a factor accounting for the expansion of the Universe. The index $s$ takes into account the fact that different types of stars can lead to supernovae with different neutrino emission spectra. One notable example is the one of neutron-star-forming supernovae vs black-hole-forming supernovae. The dependence of the outcome of the supernova and of the neutrino emission spectrum in the star's characteristics is however still poorly understood. In the code, we will consider a simplified version of equation~\ref{eq:dsnb} where this dependence can be absorbed in $F_i$:
\begin{align}
    \Phi(E_\nu) = \frac{c}{H_0}\int & R_{SN}(z)\, \sum_{\nu_i,\bar\nu_i} F_{i}\left[E_\nu (1 + z)\right] \times \frac{dz}{\sqrt{\Omega_M(1+z)^3 + \Omega_\Lambda}}.
    \label{eq:dsnb:simplified}
\end{align}
\section{The core-collapse supernova rate}
\label{sec:ccsnr}
Stars with masses between $8$ and $125 M_\odot$ end their lives in a core-collapse supernova. Lighter stars become white dwarves (which sometimes explode in Ia supernovae) and heavier stars undergo a thermonuclear explosion triggered by the $\gamma \rightarrow e^+ e^-$ process taking place in their core\footnote{This process leads to a sudden drop in the radiation pressure and hence a breakdown of the pressure-temperature relationship that was regulating the nuclear fusion processes inside the star.}. So the Core-Collapse Supernova Rate (CCSNR) can be expressed as a function of the Star Formation Rate (SFR) $R_{SF}(z)$ and the Initial Mass Function (IMF, aka the mass distribution of the stars, measured at the beginning of their lives) $\phi(m)$:
\begin{align}
    R_{SN}(z) &= R_{SF}(z)\frac{\int_{8M_\odot}^{125M_\odot}\phi(m)\,\mathrm{d}m}{\int_{0.5M_\odot}^\infty m \phi(m)\,\mathrm{d}m}.
    \label{eq:CCSNR}
\end{align}
The IMF is usually a power law $\phi(m)\propto m^{-\xi}$. The most popular one seems to be the Salpeter IMF with $\xi = 2.35$. Interestingly, the CCSNR depends only very weakly (aka percent-level) on the shape of the IMF as the latter appears both in the numerator and the denominator of equation~\ref{eq:CCSNR}. So, the CCSNR is extremely tightly related to the star formation rate. Here, I will parameterize $R_{SF}$ using a fit done in 2008 by Horiuchi, Beacom, and Dweck~\cite{Horiuchi:2008jz}. Numerous estimates of $R_{SF}$ have been used in the past to estimate the DSNB. It would be interesting to see which ones are up to date and understand whether the DSNB can give unique information about $R_{SF}$\footnote{It seems that the current exclusion bounds on the DSNB already provide an upper bound on the SFR. In 2003, Strigari and Beacom argued that the current upper limit from SK was already in tension with the current observational estimates of the SFR.}. From now on, I will write the SFR as
\begin{align}
    R_{SF}(z) = R_0 \left[(1 + z)^{\alpha\eta} + \left(\frac{1 + z}{B}\right)^{\beta\eta} + \left(\frac{1 + z}{C}\right)^{\gamma\eta}\right]^{1/\eta}.
    \label{eq:SFR}
\end{align}
The constants $B$ and $C$ are defined as
\begin{align}
    B &= (1 + z_1)^{1 - \alpha/\beta}\\
    C &= (1 + z_1)^{(\beta - \alpha)/\gamma}(1 + z_2)^{1 - \beta/\gamma}.
\end{align}
The free parameters can take the values shown in table~\ref{tab:csfrpar}.
\begin{table}[h]
    \centering
    \begin{tabular}{ccccccc}
        \toprule
        Fits & $R_0$ & $\alpha$ & $\beta$ & $\gamma$ & $z_1$ & $z_2$\\
        \midrule
        Upper & 0.0213 & 3.6 & -0.1 & -2.5 & 1 & 4\\
        Fiducial & 0.0178 & 3.4 & -0.3 & -3.5 & 1 & 4\\
        Lower & 0.0142 & 3.2 & -0.5 & -4.5 & 1 & 4\\
        \bottomrule
    \end{tabular}
    \label{tab:csfrpar}
    \caption{Fit parameters for the star formation rate using pre-2008 observations.}
\end{table}
Note that the CCSNR has also been directly measured. Right now, there is a mismatch between the observed CCSNR and the one inferred from the SFR, but this mismatch can be explained by taking the effects from dust extinction into account. Part of this effect could also be due to black-hole-forming supernovae but currently we cannot verify this. One way to extract the rate of BH forming supernovae would actually be to study the DSNB spectrum, as will be shown in the next section.
\section{Neutrino emission spectra}
\label{sec:emission}
\subsection{Blackbody or not blackbody?}
Predicting the neutrino emission spectrum from supernovae and linking it to the progenitor's characteristics is a particularly difficult problem. The only supernova neutrino spectrum ever directly observed is the one from SN1987A, and to make more generic and precise estimates we have to rely on simulations. In first approximation, neutrino emission can be modeled using a blackbody spectrum~\cite{Horiuchi:2008jz}
\begin{align}
    F_i(E_\nu) &= f_{\nu, i}E_\nu^{tot}\,\frac{120}{7\pi^4}\,\frac{E_{\nu,i}^3}{T_{\nu,i}^4}\left(e^{E_{\nu, i}/T_{\nu, i}} - 1\right)^{-1}
    \label{eq:blackbody}
\end{align}
where $f_{\nu,i}$ is the fraction of (anti)neutrinos of flavor $i$ and $E_\nu^{tot}$ is the total amount of neutrino energy emitted by a supernova. With this spectrum, in first approximation, the two parameters we can extract from the DSNB are the typical temperature of the neutrinos emitted by the supernova, the amount of energy emitted by a supernova, and the normalization of the star formation rate. Right now, we assume that all supernovae can be represented by one ``generic supernova'' but this assumption can be relaxed by defining different categories of supernovae with their own $E^{tot}$ and temperature (probably much more realistic).

One problem with the blackbody approximation is that it does not model well the proto-neutron-star (PNS) cooling (what about the accretion phase?). In fact, neutrinos coming from the deepest (and densest) regions of the collapsing star have to painstaikingly scatter out of these regions before being able to free stream. High energy neutrinos, having a higher interaction cross-section, will tend to lose more energy than their lower energy counterpart before being able to get out of the PNS. Hence, the neutrino emission spectrum should be more ``pinched'' than a Fermi-Dirac spectrum~\cite{Lunardini:2012ne}. We therefore introduce a pinching parameter alpha and choose the following more accurate formulation for the neutrino flux (dropping the ``flavor'' index $i$):
\begin{align}
    F(E_\nu) = \frac{L \,.\, (1 + \alpha)^{1 + \alpha}}{\Gamma(1 + \alpha)\langle E_\nu\rangle^2}\left(\frac{E}{\langle E_\nu \rangle}\right)^\alpha \exp\left[-(1 + \alpha)\frac{E}{\langle E_\nu \rangle}\right].
    \label{eq:pinched}
\end{align}
Here $L$ is the total luminosity for the type of neutrino considered.
\subsection{Flavor oscillations}
When considering neutrino emission, we also have to take flavor oscillation into account. We can consider that flavor conversion will come from two effects:
\begin{itemize}
    \item Scattering processes: $\nu_i \bar{\nu}_i\rightarrow \nu_j \bar{\nu}_j$
    \item MSW effect
\end{itemize}
Flavor conversion due to scattering processes is particularly difficult to model and requires an in-depth knowledge of the dynamics of the core of the star during the supernova. Right now, it seems that these processes can lead to parts of the neutrino energy spectra being ``swapped'' between the different flavors. This swapping gives neutrino spectra with sharp features called spectral splits. The impact of these processes on the DSNB spectrum is uncertain. It is possible that their effects average out due to the integration over all supernovae but there have not been any reliable predictions yet. It seems that most papers assume scattering effects are washed out and focus on MSW flavor conversion~\cite{Lunardini:2012ne}. MSW effects can be modeled by assuming two flavors of neutrinos, $e$ and $x$, and introducing survival probabilities for electron neutrinos and antineutrinos. Assuming adiabatic conversion, the values for these probabilities will depend on the mass hierarchy. This way, we can write the oscillated fluxes as:
\begin{align}
    F_{\bar{\nu}_e} &\rightarrow \bar{p} F_{\bar{\nu}_e} + (1 - \bar{p}) F_{\bar{\nu}_x}\quad\quad \bar{p} = 0\text{ (IH) or }0.68\text{ (NH)}\\
    F_{\nu_e} &\rightarrow p F_{\nu_e} + (1 - p) F_{\nu_x}\quad\quad p = 0.68\text{ (IH) or }0.33\text{ (NH)}.
    \label{eq:flavor}
\end{align}
\subsection{Dependence in the type of supernova}
Although we know that assuming one generic supernova is wrong, categorizing supernovae is a difficult problem since we have to rely on simulations. Generally, supernovae are divided into two categories: ``successful'' supernovae, that lead to an explosion and leave a neutron star (NS) behind, and ``failed'' supernovae, that do not explode and just collapse into a black hole (BH). The latter generally happen to very heavy (or compact?) stars where the shock wave generated by the initial collapse cannot travel out of the star's core, even with the help of neutrinos. The neutrinos emitted are typically more energetic since the collapse process is more violent. Hence, studying the tails of the DSNB could help us estimate the fraction of these supernovae. This fraction is currently not known, as no such supernova has been directly observed yet (current upper bound is about $44$\%).  

The simplest way to parameterize the impact of BH-forming supernovae on the DSNB is to introduce the BH fraction $f_{BH}$ and rewrite the DSNB as:
    \begin{align}
        \Phi(E_\nu) = \frac{c}{H_0}\int & R_{SN}(z)\,\left[F_{\nu, NS}(E_\nu (1 + z)) + f_{BH}F_{\nu, BH}(E_\nu (1 + z))\right]\\
    & \times \frac{dz}{\sqrt{\Omega_M(1+z)^3 + \Omega_\Lambda}}
    \label{eq:BH1}
    \end{align}
    where the NS and BH neutrino fluxes have their own luminosities, mean energies, and pinching parameters. This is the parameterization currently used in the spectrum generation code.

    One issue with equation~\ref{eq:BH1} is that $f_{BH}$ should strongly depend on the progenitor star's characteristics. Therefore, we cannot separate it from $R_{SN}$. In~\cite{Priya:2017bmm}, Priya and Lunardini suggested to use the star's mass and its metallicity to define mass ranges in which all supernovae form BH. They also took into account the dependence of the neutrino spectrum in the progenitor's mass, even for supernovae with the same outcome (using simulations for a few chosen masses and assuming a step function for the mass dependence). They thus used the following formula for the DSNB: 
    \begin{align}
        \Phi(E_\nu) = \frac{c}{H_0\int_{0.5M_\odot}^\infty m \phi(m)\,\mathrm{d}m}\int & R_{SF}(z)\int_{8M_\odot}^{125M_\odot}\mathrm{d}m\,\phi(m)\,F_{\nu}(E_\nu (1 + z), m) \\
        &\times \frac{dz}{\sqrt{\Omega_M(1+z)^3 + \Omega_\Lambda}}
    \label{eq:BH1}
    \end{align}
    Mass and metallicity as however not very good predictor of a supernova outcome in simulation. Recently, a new (better) predictor has been proposed: the star's compactness $\xi$, defined as
    \begin{align}
    \xi = \left|\frac{M/M_\odot}{R(M)/1000~\mathrm{km}}\right|_t
    \end{align}
    where $M$ is a mass scale, and $R(M)$ is the radius enclosing $M$ at epoch $t$. Generally, $t$ is taken to be right before the collapse and $M = 1.5-2.5M_\odot$. With this parameterization, the shape of the DSNB spectrum could give the ``critical compactness'' $\xi_c$ beyond which supernovae form black holes. Using the value of $\xi_c$ and observational information, one can thus find the fraction of BH forming supernovae, but also better understand how the star's properties influence the supernova outcome~\cite{Horiuchi:2017qja}.
    \section{The code}
    The code tries to combine simple approximations found in the literature, notably:
    \begin{itemize}
        \item Fitted star formation rate from equation~\ref{eq:SFR}
        \item Blackbody and pinched spectra from equations~\ref{eq:blackbody} and \ref{eq:pinched}
        \item MSW flavor effects from equation~\ref{eq:flavor}
        \item Black hole fraction from equation~\ref{eq:BH1}
    \end{itemize}
    The interface is in Python 3 but many of the underlying functions are written in C++11. If you want to modify these ones and continue using the Python interface, you will need to install Pybind 11. If you just want to use the code and have Linux 64 bits, you can just copy my library. You will need to have Python 3 + Numpy + Scipy installed though. The wrapper functions are in \texttt{snspectrum.py}.

    One example of how to use the code is the function \texttt{param\_scan}, at the very end of \texttt{snspectrum.py}. The neutrino spectrum, IMF, and SFR parameters are stored in dictionaries. You can directly modify the values in the code. The function \texttt{ibd\_spectrum} computes the antineutrino spectrum that would be observed in SK without detector effects. \texttt{ibd\_spectrum} has to be called for each value of the positron energy and returns a number of events.
    \begin{thebibliography}{}
        %\cite{Horiuchi:2008jz}
        \bibitem{Horiuchi:2008jz} 
          S.~Horiuchi, J.~F.~Beacom and E.~Dwek,
          %``The Diffuse Supernova Neutrino Background is detectable in Super-Kamiokande,''
          Phys.\ Rev.\ D {\bf 79}, 083013 (2009)
          doi:10.1103/PhysRevD.79.083013
          [arXiv:0812.3157 [astro-ph]].
          %%CITATION = doi:10.1103/PhysRevD.79.083013;%%
          %142 citations counted in INSPIRE as of 11 Dec 2019
        %\cite{Priya:2017bmm}
        \bibitem{Priya:2017bmm} 
          A.~Priya and C.~Lunardini,
          %``Diffuse neutrinos from luminous and dark supernovae: prospects for upcoming detectors at the $O$(10) kt scale,''
          JCAP {\bf 1711}, 031 (2017)
          doi:10.1088/1475-7516/2017/11/031
          [arXiv:1705.02122 [astro-ph.HE]].
          %%CITATION = doi:10.1088/1475-7516/2017/11/031;%%
          %7 citations counted in INSPIRE as of 11 Dec 2019
        %\cite{Lunardini:2012ne}
        \bibitem{Lunardini:2012ne} 
          C.~Lunardini and I.~Tamborra,
          %``Diffuse supernova neutrinos: oscillation effects, stellar cooling and progenitor mass dependence,''
          JCAP {\bf 1207}, 012 (2012)
          doi:10.1088/1475-7516/2012/07/012
          [arXiv:1205.6292 [astro-ph.SR]].
          %%CITATION = doi:10.1088/1475-7516/2012/07/012;%%
          %24 citations counted in INSPIRE as of 11 Dec 2019
        %\cite{Horiuchi:2017qja}
        \bibitem{Horiuchi:2017qja} 
          S.~Horiuchi, K.~Sumiyoshi, K.~Nakamura, T.~Fischer, A.~Summa, T.~Takiwaki, H.~T.~Janka and K.~Kotake,
          %``Diffuse supernova neutrino background from extensive core-collapse simulations of $8$-$100 {\rm M}_\odot$ progenitors,''
          Mon.\ Not.\ Roy.\ Astron.\ Soc.\  {\bf 475}, no. 1, 1363 (2018)
          doi:10.1093/mnras/stx3271
          [arXiv:1709.06567 [astro-ph.HE]].
          %%CITATION = doi:10.1093/mnras/stx3271;%%
          %18 citations counted in INSPIRE as of 11 Dec 2019
    \end{thebibliography}
\end{document}
